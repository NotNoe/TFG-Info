\chapter*{Resumen}

\section*{\tituloPortadaVal}

Este trabajo se centra en el uso de técnicas de Inteligencia Artificial para la clasificación de señales de electrocardiogramas (ECGs), un área de gran relevancia en la medicina para el diagnóstico de enfermedades cadíacas. Este trabajo tiene dos objetivos principales: explorar y evaluar diferentes enfoques de clasificación de ECG y aplicar explicabilidad a algunos de los modelos entrenados.

Para el primer objetivo, utilizamos bases de datos abiertas como PTB-XL, y aplicar transformaciones de señales como la Transformada de Fourier de Tiempo Reducido (STFT) o la Transformada de Onda Continua (CWT). Con esto entrenaremos y compararemos modelos clásicos y modificados para trabajar con estas transformadas, y se analiza su rendimiento usando métricas.

La explicabilidad de los modelos es también un aspecto clave del estudio, ya puede emplearse tanto para docencia como para justificar las decisiones tomadas por el modelo. Al final de este trabajo validamos los resultados con un experto médico para asegurar la aplicabilidad clínica y docente de los resultados obtenidos.

Las conclusiones obtenidas subrayan la importancia de la colaboración entre la IA y los profesionales de la salud. Por último se proponen direcciones futuras para la mejora y expansión del enfoque propuesto.

\section*{Palabras clave}
   
ECG, predicción de anomalías, STFT, CWT, explicabilidad en la IA, redes neuronales profundas, clasificación multietiqueta, riesgo cardiovascular.
   



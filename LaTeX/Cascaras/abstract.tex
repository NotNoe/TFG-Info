\chapter*{Abstract}

\section*{\tituloPortadaEngVal}

This work focuses on the use of Artificial Intelligence techniques for the classification of electrocardiogram (ECG) signals, a highly relevant area in medicine for the diagnosis of cardiac diseases. The study has two main objectives: to explore and evaluate different approaches for ECG classification and to apply explainability to some of the trained models.

For the first objective, we use open databases such as PTB-XL and apply signal transformations like the Short-Time Fourier Transform (STFT) and Continuous Wavelet Transform (CWT). These transformations are used to train and compare classic models and modified models designed to work with these transformations, analyzing their performance using metrics.

Model explainability is also a key aspect of the study, as it can be employed both for educational purposes and to justify decisions made by the model. At the end of this work, the results are validated by a medical expert to ensure their clinical and educational applicability.

The findings highlight the importance of collaboration between AI and healthcare professionals. Finally, future directions are proposed for improving and expanding the proposed approach.

\section*{Keywords}

ECG, anomaly detection, STFT, CWT, AI explainability, deep neural networks, multilabel classification, cardiovascular risk.


\chapter{Conclusiones y Trabajo Futuro}
\label{cap:conclusiones}

\section{Conclusiones}
A lo largo de este trabajo hemos demostrado que el modelo de Ribeiro funciona de manera adecuada aún entrenado con una base de datos más pequeña, lo que señala que es una arquitectura bastante sólida y prometedora.

No obstante, a la hora de introducir transformaciones, hemos descubierto que la arquitectura del modelo de Ribeiro no es adecuada para manejar imágenes, como es el caso de las transformadas. Esto no significa que el modelo de Ribeiro no pueda ser adaptado para trabajar con imágenes de manera adecuada, ya que una arquitectura con unos resultados tan buenos es muy prometedora.

Es importante señalar que el campo del \emph{deep learning} aplicado a la medicina es relativamente nuevo, y se publican trabajos innovadores y avances muy frecuentemente, por lo que el uso de transformadas para el análisis de ECGs sigue siendo una posible dirección de avance.

En cuanto a la explicabilidad, los \emph{saliency maps} basados en gradientes han mostrado ser un recurso prometedor para entender qué partes de la señal del ECG resultan relevantes para la red. Refinar estas explicaciones podría convertirlas en una herramienta de aprendizaje y educación útil tanto para estudiantes como para profesionales sanitarios que busquen familiarizarse con la IA. Además hemos visto como los profesionales reaccionaron muy positivamente a la existencia de herramientas que permitan explicar las decisiones de los modelos, y están convencidos de que, una vez se mejoren, van a ser un recurso muy importante en la educación.

\section{Trabajo futuro}
Dado el potencial del modelo y las limitaciones detectadas, se plantean varias líneas de trabajo para mejorar y ampliar los resultados:
\begin{enumerate}
	\item \textbf{Modificar la arquitectura para admitir capas Conv2D} \\
	Esta adaptación, si bien requeriría de un trabajo significativo, permitiría explotar mejor las ventajas de las transformaciones de la señal.
	\item \textbf{Explorar transformadas y parámetros adicionales} \\
	Siguiendo la línea anterior, una vez obtenida una arquitectura más adecuada para imágenes, podrían probarse otros parámetros a la hora de hacer las transformadas y estudiar su impacto.
	\item \textbf{Especializar el modelo en una anomalía} \\
	Transformar el modelo en un clasificador binario permitiría elegir los parámetros de las transformadas de manera más precisa, ya que distintas anomalías se manifiestan en distintas frecuencias.
	\item \textbf{Cambiar los umbrales de detección} \\
	En el trabajo de Ribeiro, en lugar de utilizar el umbral de 0.5 para clasificar las etiquetas, se calcularon los umbrales que optimizaban los resultados sobre el conjunto de validación. Esto es algo que, si bien en este trabajo no hemos abordado, podría mejorar los resultados de todos los modelos que hemos entrenado.
	\item \textbf{Delineación de señales en la explicación} \\
	Como señaló el experto, mediante delineación de señales podrían reducirse las explicaciones a dos latidos y asegurarse de que estas son homogéneas en ambos latidos. Estas mejoras harían viable el uso de estas explicaciones en el ámbito docente.
\end{enumerate}


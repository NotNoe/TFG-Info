\chapter*{Conclusions and Future Work}
\label{cap:conclusions}
\addcontentsline{toc}{chapter}{Conclusions and Future Work}
\section{Conclusions}
Over the course of this work, the applicability of Ribeiro’s model for classifying ECG's from the PTB-XL database has been demonstrated, achieving acceptable results when processing the original signal. However, when transformations (STFT, CWT, etc.) are introduced and the same one-dimensional convolution-based architecture is maintained, the results deteriorate significantly. This observation suggests that Ribeiro’s architecture, in its current form, is not suitable for handling images, thus requiring substantial modifications to fully exploit the time-frequency representations offered by the transforms.

Regarding explainability, gradient-based \emph{saliency maps} have proven to be a promising resource for understanding which parts of the ECG signal are relevant to the network. Refining these explanations could turn them into a valuable learning and educational tool for both students and healthcare professionals seeking to familiarize themselves with AI.

\section{Future Work}
Given the model’s potential and the identified limitations, various lines of work are proposed to improve and extend the results:
\begin{enumerate}
	\item \textbf{Modify the architecture to support Conv2D layers} \\
	Although this adaptation would require significant effort, it would allow for better exploitation of the advantages of signal transformations.
	\item \textbf{Explore additional transforms and parameters} \\
	Following the previous line, once an architecture more suitable for images is obtained, other parameters could be tested when performing the transforms, and their impact could be studied.
	\item \textbf{Specialize the model in one anomaly} \\
	Transforming the model into a binary classifier would allow for more precise parameter selection for the transforms, since different anomalies manifest in different frequency ranges.
	\item \textbf{Signal delineation in explanations} \\
	As noted by the expert, through signal delineation it would be possible to reduce explanations to two heartbeats and ensure they are homogeneous in both. These improvements would make the use of these explanations viable in educational settings.
\end{enumerate}

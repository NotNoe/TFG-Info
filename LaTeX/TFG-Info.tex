% ----------------------------------------------------------------------
%
%                            TFMTesis.tex
%
%----------------------------------------------------------------------
%
% Este fichero contiene el "documento maestro" del documento. Lo único
% que hace es configurar el entorno LaTeX e incluir los ficheros .tex
% que contienen cada sección.
%
%----------------------------------------------------------------------
% rm -rf *.idx *.lof *.log *.out *.synctex.gz *.toc *.bst *.aux *.bbl
% Los ficheros necesarios para este documento son:
%
%       TeXiS/* : ficheros de la plantilla TeXiS.
%       Cascaras/* : ficheros con las partes del documento que no
%          son capítulos ni apéndices (portada, agradecimientos, etc.)
%       Capitulos/*.tex : capítulos de la tesis
%       Apendices/*.tex: apéndices de la tesis
%       constantes.tex: constantes LaTeX
%       config.tex : configuración de la "compilación" del documento
%       guionado.tex : palabras con guiones
%
% Para la bibliografía, además, se necesitan:
%
%       *.bib : ficheros con la información de las referencias
%
% ---------------------------------------------------------------------

\documentclass[12pt,a4paper,twoside]{book}
\interfootnotelinepenalty=10000

%
% Definimos  el   comando  \compilaCapitulo,  que   luego  se  utiliza
% (opcionalmente) en config.tex. Quedaría  mejor si también se definiera
% en  ese fichero,  pero por  el modo  en el  que funciona  eso  no es
% posible. Puedes consultar la documentación de ese fichero para tener
% más  información. Definimos también  \compilaApendice, que  tiene el
% mismo  cometido, pero  que se  utiliza para  compilar  únicamente un
% apéndice.
%
%
% Si  queremos   compilar  solo   una  parte  del   documento  podemos
% especificar mediante  \includeonly{...} qué ficheros  son los únicos
% que queremos  que se incluyan.  Esto  es útil por  ejemplo para sólo
% compilar un capítulo.
%
% El problema es que todos aquellos  ficheros que NO estén en la lista
% NO   se  incluirán...  y   eso  también   afecta  a   ficheros  de
% la plantilla...
%
% Total,  que definimos  una constante  con los  ficheros  que siempre
% vamos a querer compilar  (aquellos relacionados con configuración) y
% luego definimos \compilaCapitulo.
\newcommand{\ficherosBasicosTeXiS}{%
TeXiS/TeXiS_pream,TeXiS/TeXiS_cab,TeXiS/TeXiS_bib,TeXiS/TeXiS_cover%
}
\newcommand{\ficherosBasicosTexto}{%
constantes,guionado,Cascaras/bibliografia,config%
}
\newcommand{\compilaCapitulo}[1]{%
\includeonly{\ficherosBasicosTeXiS,\ficherosBasicosTexto,Capitulos/#1}%
}

\newcommand{\compilaApendice}[1]{%
\includeonly{\ficherosBasicosTeXiS,\ficherosBasicosTexto,Apendices/#1}%
}

%- - - - - - - - - - - - - - - - - - - - - - - - - - - - - - - - - - -
%            Preámbulo del documento. Configuraciones varias
%- - - - - - - - - - - - - - - - - - - - - - - - - - - - - - - - - - -

% Define  el  tipo  de  compilación que  estamos  haciendo.   Contiene
% definiciones  de  constantes que  cambian  el  comportamiento de  la
% compilación. Debe incluirse antes del paquete TeXiS/TeXiS.sty
\include{config}

% Paquete de la plantilla
\usepackage{./TeXiS/TeXiS}

%PAQUETES PERSONALIZADOS
\usepackage{multicol}
\usepackage{graphicx}
\usepackage{threeparttable}
\usepackage{hyperref}
\usepackage{amsmath}
\usepackage{titletoc}
\usepackage{caption}


\usepackage{booktabs} % Para tablas más bonitas
\usepackage{multirow} % Para combinar filas
\usepackage{longtable}
\usepackage{listings}
\usepackage{xcolor}

\lstset{
	language=Python,                 % Lenguaje del código
	basicstyle=\ttfamily\footnotesize, % Estilo básico (tipo monoespaciado)
	keywordstyle=\color{blue},       % Color para palabras clave
	stringstyle=\color{orange},      % Color para cadenas
	commentstyle=\color{green!60!black}, % Color para comentarios
	numbers=left,                    % Numeración a la izquierda
	numberstyle=\tiny\color{gray},   % Estilo de números de línea
	stepnumber=1,                    % Cada línea numerada
	frame=single,                    % Marco alrededor del código
	breaklines=true,                 % Permitir líneas largas divididas
	captionpos=b,                    % Posición de la leyenda (abajo)
	showspaces=false,                % No mostrar espacios
	showstringspaces=false,          % No mostrar espacios en cadenas
	tabsize=4,                       % Tamaño de tabulación
	literate=                        % Configurar caracteres especiales
		{á}{{\'a}}1 {é}{{\'e}}1 {í}{{\'i}}1 {ó}{{\'o}}1 {ú}{{\'u}}1
		{Á}{{\'A}}1 {É}{{\'E}}1 {Í}{{\'I}}1 {Ó}{{\'O}}1 {Ú}{{\'U}}1
		{ñ}{{\~n}}1 {Ñ}{{\~N}}1 {¡}{{\textexclamdown}}1 {¿}{{\textquestiondown}}1
		{±}{{$\pm$}}1                
}

% Incluimos el fichero con comandos de constantes
\include{constantes}

% Sacamos en el log de la compilación el copyright
%\typeout{Copyright Marco Antonio and Pedro Pablo Gomez Martin}

%
% "Metadatos" para el PDF
%
\ifpdf\hypersetup{%
    pdftitle = {\titulo},
    pdfsubject = {Trabajo de Fin de Grado de Noelia Barranco Godoy},
    pdfkeywords = {Redes neuronales, ECG, TFG},
    pdfauthor = {\textcopyright\ \autor},
    pdfcreator = {\LaTeX\ con el paquete \flqq hyperref\frqq},
    pdfproducer = {pdfeTeX-0.\the\pdftexversion\pdftexrevision},
    }
    \pdfinfo{/CreationDate (\today)}
\fi


%- - - - - - - - - - - - - - - - - - - - - - - - - - - - - - - - - - -
%                        Documento
%- - - - - - - - - - - - - - - - - - - - - - - - - - - - - - - - - - -
\begin{document}

% Incluimos el  fichero de definición de guionado  de algunas palabras
% que LaTeX no ha dividido como debería
\input{guionado}

% Marcamos  el inicio  del  documento para  la  numeración de  páginas
% (usando números romanos para esta primera fase).
\frontmatter
\pagestyle{empty}

\include{Cascaras/cover}
% +--------------------------------------------------------------------+
% | Dedication Page (Optional)
% +--------------------------------------------------------------------+

\chapter*{Dedicatoria}

\begin{flushright}
\begin{minipage}[c]{8.5cm}
\flushright{\it
A mi pareja, Ian, por ser mi refugio en cada instante de agobio y recordarme, con su presencia y cariño, que ninguna meta es inalcanzable cuando no se afronta sola.}
\end{minipage}
\end{flushright}
% +--------------------------------------------------------------------+
% | Acknowledgements Page (Optional)                                   |
% +--------------------------------------------------------------------+

\chapter*{Agradecimientos}

Deseo expresar mi más sincero reconocimiento a los directores de mi TFG, en particular a la profesora Belén, por su constante apoyo y por haberme presentado a Marian. Su presencia, más allá de los aspectos estrictamente académicos, fue esencial a nivel profesional y personal, ofreciéndome una dedicación, cercanía y orientación que marcaron profundamente mi proceso formativo. Estas atenciones, que superaron con creces el ámbito académico, han dejado una huella imborrable en mi trayectoria.





\chapter*{Resumen}

\section*{\tituloPortadaVal}

Este trabajo se centra en el uso de técnicas de Inteligencia Artificial para la clasificación de señales de electrocardiogramas (ECGs), un área de gran relevancia en la medicina para el diagnóstico de enfermedades cadíacas. Este trabajo tiene dos objetivos principales: explorar y evaluar diferentes enfoques de clasificación de ECG y aplicar explicabilidad a algunos de los modelos entrenados.

Para el primer objetivo, utilizamos bases de datos abiertas como PTB-XL, y aplicar transformaciones de señales como la Transformada de Fourier de Tiempo Reducido (STFT) o la Transformada de Onda Continua (CWT). Con esto entrenaremos y compararemos modelos clásicos y modificados para trabajar con estas transformadas, y se analiza su rendimiento usando métricas.

La explicabilidad de los modelos es también un aspecto clave del estudio, ya puede emplearse tanto para docencia como para justificar las decisiones tomadas por el modelo. Al final de este trabajo validamos los resultados con un experto médico para asegurar la aplicabilidad clínica y docente de los resultados obtenidos.

Las conclusiones obtenidas subrayan la importancia de la colaboración entre la IA y los profesionales de la salud. Por último se proponen direcciones futuras para la mejora y expansión del enfoque propuesto.

\section*{Palabras clave}
   
ECG, predicción de anomalías, STFT, CWT, explicabilidad en la IA, redes neuronales profundas, clasificación multietiqueta, riesgo cardiovascular.
   



\begin{otherlanguage}{english}
\chapter*{Abstract}

\section*{\tituloPortadaEngVal}

This work focuses on the use of Artificial Intelligence techniques for the classification of electrocardiogram (ECG) signals, a highly relevant area in medicine for the diagnosis of cardiac diseases. The study has two main objectives: to explore and evaluate different approaches for ECG classification and to apply explainability to some of the trained models.

For the first objective, we use open databases such as PTB-XL and apply signal transformations like the Short-Time Fourier Transform (STFT) and Continuous Wavelet Transform (CWT). These transformations are used to train and compare classic models and modified models designed to work with these transformations, analyzing their performance using metrics.

Model explainability is also a key aspect of the study, as it can be employed both for educational purposes and to justify decisions made by the model. At the end of this work, the results are validated by a medical expert to ensure their clinical and educational applicability.

The findings highlight the importance of collaboration between AI and healthcare professionals. Finally, future directions are proposed for improving and expanding the proposed approach.

\section*{Keywords}

ECG, anomaly detection, STFT, CWT, AI explainability, deep neural networks, multilabel classification, cardiovascular risk.


\end{otherlanguage}

\ifx\generatoc\undefined
\else
\include{TeXiS/TeXiS_toc}
\fi

% Marcamos el  comienzo de  los capítulos (para  la numeración  de las
% páginas) y ponemos la cabecera normal
\mainmatter

\pagestyle{fancy}
\restauraCabecera


\chapter{Introducción}
\label{cap:introduccion}
\begin{resumen}
	En este capítulo abordaremos la relevancia de la IA en el ámbito médico, describiremos los objetivos y el alcance del trabajo, y presentaremos la estructura general del documento.
\end{resumen}

\section{Fundamentos del electrocardiograma}
Un electrocardiograma (ECG) es una prueba diagnóstica no invasiva que registra la actividad eléctrica del corazón a lo largo del tiempo. Mediante la colocación de electrodos en puntos específicos del cuerpo (usualmente diez), se captan variaciones en los potenciales eléctricos producidos por el músculo cardíaco \citep{ribeiro}. Estas variaciones se reflejan en la forma de ondas y complejos (P, QRS, T), cuya interpretación permite identificar diversas patologías. Aunque estas mediciones pueden hacerse de diversas formas, lo habitual es medir la diferencia de potencial entre doce pares de electrodos. Cada una de estas mediciones genera una señal, que recibe el nombre de derivación.

Dado que no es posible recoger una cantidad continua de mediciones en el tiempo, se toma una cantidad de mediciones finita a una frecuencia específica (usualmente de 100Hz a 500Hz, dependiendo de la precisión del aparato medidor). Por tanto un ECG puede almacenarse en doce vectores de igual tamaño (variable dependiendo de la duración y frecuencia de la prueba). En la Figura \ref{fig:ecg} podemos ver un ejemplo de electrocardiograma (extraido de \cite{ptbxldb} y dibujado mediante la librería de python \emph{\text{ecg\_plot}}).

\begin{figure}[t]
	\setlength{\fboxsep}{0pt}%
	\fbox{\includegraphics[width=\linewidth]{../Codigo/out/ecg100.png}}
	\fbox{\includegraphics[width=\linewidth]{../Codigo/out/ecg500.png}}
	\caption{ECG de 12 derivaciones tomado a 100Hz (arriba) y 500Hz (abajo).}
	\label{fig:ecg}
\end{figure}

La frecuencia a la que se mide un ECG es muy importante. Si tiene una mayor frecuencia, la onda se puede ver con una mayor resolución, lo que permite detectar mejor las posibles anomalías. No obstante, además de que un aparato con capacidad de medir a más resolución es más complejo y caro, hay otras desventajas de tener una frecuencia muy alta. Una mayor frecuencia implica que para almacenar un ECG de la misma duración necesitamos más memoria, y será computacionalmente más costoso cualquier tipo de procesamiento o análisis que queramos realizar sobre este.

La relevancia clínica del ECG es enorme, además de ser una prueba de bajo coste y gran disponibilidad, el electrocardiograma constituye el primer paso para la detección de anomalías cardíacas en servicios de urgencias, consultas de atención primaria y especializadas. De acuerdo con la Organización Mundial de la Salud (OMS), las enfermedades cardiovasculares representan una de las principales causas de mortalidad a nivel global \citep{whocvd}; por ello, optimizar su diagnóstico temprano a través de métodos automatizados de análisis y clasificación puede tener un impacto significativo en la mejora de la salud pública.

En este contexto, se han desarrollado modelos de \emph{Deep Learning} muy precisos para el diagnóstico interpretable de anomalías cardíacas \citep{lu2024decoding}. A lo largo de este capítulo se profundizará en aspectos esenciales de la señal, se expondrán los fundamentos de las ondas principales y se describirán las bases de datos de ECGs más utilizadas para la investigación.

\subsection{Ondas y segmentos principales}

En la Figura \ref{fig:latido} se muestra un trazado típico de un ECG con etiquetas para cada onda y segmento. A grandes rasgos se pueden distinguir:
\begin{itemize}
	\item \textbf{Onda P}: la primera elevación relativamente pequeña.
	\item  \textbf{Complejo QRS}: El conjunto de picos y valles centrales, normalmente lo más destacado.
	\item \textbf{Onda T}: La elevación posterior que aparece después del complejo QRS.
	\item \textbf{Segmento ST}: La sección entre el final del complejo QRS y el inicio de la onda T.
	\item \textbf{Intervalo PR}: El espacio desde el inicio de la onda P hasta el comienzo del complejo QRS.
	\item \textbf{Intervalo QT}: El espacio desde el inicio de la onda Q hasta el final de la onda T.
	\item \textbf{Segmento PR}: La sección entre el final de la onda P y el comiendo del complejo QRS.
\end{itemize}

\begin{figure}
	\centering
	\includegraphics[width=0.5\textwidth]{Imagenes/Vectorial/latido.png}
	\caption{Ejemplo de ECG con etiquetas en P, QRS, T, ST y PR. Fuente: \cite{latido_img}}
	\label{fig:latido}
\end{figure}

Aunque cada uno de estos componentes tiene implicaciones fisiológicas y diagnósticas, en este trabajo no entraremos en ellos, ya que el conocimiento profundo de los procesos cardíacos que generan cada onda no es necesario para las técnicas de análisis y clasificación que utilizaremos. Para una descripción en detalle de las partes de un ECG, así como de la relación con los movimientos del músculo cardíaco, el lector puede referirse a \textbf{Manual de Electrocardiografía Básica} \cite{manualelectro}.

\subsection{Principales anomalías en un ECG}

Existen numerosas anomalías en un ECG que pueden reflejar alteraciones en la función cardíaca. En este trabajo, agruparemos las anomalías de la misma forma en la que están agrupadas en la base de datos elegida, que podemos ver en la Sección \ref{subsec:anomalias}.


\section{Motivación}
La Inteligencia Artificial (IA) ha transformado numerosos campos. En particular, la medicina ha experimentado avances significativos mediante la aplicación de la inteligencia artificial al diagnóstico y análisis de datos. Uno de los campos más prometedores es el de la interpretación automatizada de señales en un ECG, fundamentales para la detección temprana de enfermedades cardíacas. Las enfermedades cardiovasculares son una de las principales causas de muerte a nivel global \citep{whocvd}, lo que subraya la necesidad de métodos rápidos y precisos para su diagnóstico.

El análisis de ECGs ha sido históricamente una tarea realizada por profesionales de la salud, como cardiólogos, debido a la complejidad y variabilidad de las señales y de la interpretación de estas. Sin embargo, este proceso es lento, costoso y depende en gran medida de la experiencia del especialista. Con la adopción de modelos de \emph{Deep Learning}, como las redes neuronales convolucionales (CNN), surge la oportunidad de automatizar y mejorar este análisis, proporcionando resultados rápidos, y en muchos casos, comparables a los obtenidos por expertos humanos \citep{hannun_cardiologist_2019}. Este enfoque no solo promete aumentar la eficiencia del diagnóstico, sino también mejorar la precisión y reducir la carga de trabajo del personal médico.

Sin embargo, los médicos se han mostrado reacios a utilizar estas tecnologías debido a que en muchas ocasiones, por ser modelos predictivos que no proporcionan explicaciones (modelos de caja negra), no pueden comprender su funcionamiento. Para que este tipo de herramientas genere confianza en entornos clínicos y cumpla con los estándares de calidad y ética, es fundamental que sean explicables \citep{Molnar2019}. La explicabilidad de los modelos de IA hace posible comprender y justificar las decisiones tomadas durante el análisis de la señal cardíaca, un factor crítico en aplicaciones de alto impacto como la medicina, donde un error puede afectar directamente a la salud de los pacientes \citep{Goodman2017}.

Si bien se ha realizado una cantidad significativa de investigación en el campo de análisis y predicción de riesgos a partir de ECGs (por ejemplo \cite{Ribeiro}), no son tantos los trabajos que hay que ponen el foco en tener modelos explicables, lo que es muy importante (como señalábamos en el párrafo anterior) para que estos modelos sean realmente utilizados. Creemos que, si los modelos transformados obtienen resultados similares (o mejores) que el modelo original, estos podrían dar mejores resultados a la hora de aplicar técnicas de explicaión.

\com{Esto lo he añadido para resaltar el motivo de utilizar transformadas, ver si dan mejores explicaciones. pero claro, entre que el modelo no da buenos resultados con las transformadas y que hay limitaciones técnicas que descubrimos más adelante, no hemos llegado a probar explicar las transformadas. Esta es la última revisión del trabajo, así que creo que o dejo este último párrafo o lo quito, pero no quiero estar haciendo cambios de redacción mayores (salvo erratas o meteduras de pata similares) a estas alturas.}

A lo largo de este trabajo, exploraremos diversas arquitecturas de redes neuronales y técnicas de transformación de señales para clasificar y detectar anomalías en ECGs, con el objetivo de desarrollar y mejorar un modelo clasificador de ECGs en cuatro grupos de anomalías cardíacas. Además, incorporaremos métodos de explicabilidad para que los resultados del modelo puedan ser interpretados y validados por profesionales de la salud, lo que es esencial para la adopción clínica de estas tecnologías.

\section{Objetivos}
El principal objetivo de este trabajo es modificar, entrenar y evaluar el modelo propuesto originalmente por \cite{ribeiro} (en adelante, `\emph{gold standard}'), aplicándolo sobre una base de datos pública. Para lograrlo, introduciremos una modificacion en la capa de entrada del modelo que permita la inclusión de transformadas (por ejemplo STFT o CWT) a fin de explorar si la conversión de la señal en distintas representaciones puede mejorar el rendimiento. No obstante, no se realizan cambios en la arquitectura interna del modelo.

Con base en lo anterior, los objetivos específicos son:

\begin{enumerate}
	\item \textbf{Entrenar el \emph{gold standard}} con una base de datos pública.
	\item \textbf{Modificar el \emph{gold standard}} para que admita transformaciones de la señal, posibilitando el entrenamiento de versiones alternativas del modelo con datos transformados.
	\item \textbf{Comparar el rendimiento} de cada variante con el \emph{gold standard} mediante métricas como F1-Score, analizando si las transformaciones resultan o no beneficiosas.
	\item \textbf{Aplicar un método de explicabilidad} para que un especialista pueda comprender qué partes de la señal de un ECG influyen en la predicción.
\end{enumerate}

\section{Estructura del documento}
El presente documento se organiza en seis capítulos, cuyo contenido se describe a continuación:
\begin{itemize}
	\item \textbf{Capítulo 2: Estado del arte} \\
	Se exponen los conceptos básicos sobre electrocardiogramas, resaltando la importancia de sus ondas (P, QRS, T) y las anomalías más comunes que se suelen detectar. Asimismo, se revisa la literatura relacionada con el uso de redes neuronales en el análisis de ECGs y se describen bases de datos relevantes (como PTB-XL).
	
	\item  \textbf{Capítulo 3: Metodología y preparación de datos} \\
	Se detalla el proceso de tratamiento de los ECGs, incluyendo la división en conjuntos de entrenamiento, validación y prueba. Además, se describen las métricas utilizadas para evaluar los modelos y las transformaciones empleadas.
	
	\item \textbf{Capítulo 4: Entrenamiento y resultados}\\
	Se explican las condiciones de entrenamiento de cada modelo (incluyendo las modificaciones del \emph{gold standard}) y las librerías utilizadas y se presentan y comparan los resultados cuantitativos obtenidos por las métricas.
	
	\item \textbf{Capítulo 5: Explicabilidad}\\
	Se describe el método de explicabilidad basado en \emph{saliency maps} de gradientes aplicado al modelo de Ribeiro. Asimismo, se incluye la perspectiva de un médico especialista que analiza las explicaciones generadas y se reflexiona sobre el método y posibles mejoras.
	
	\item \textbf{Capítulo 6: Conclusiones y trabajo futuro}\\
	Se integran las conclusiones derivadas de los resultados, valorando en qué medida se han cumplido los objetivos planteados. Además, se señalan las principales contribuciones de este trabajo y se proponen líneas de investigación futuras (como la inclusión de capas Conv2D).
\end{itemize}

Por último, se incluye un apartado de bibliografía, donde se recogen todas las fuentes consultadas a lo largo del documento, así como un anexo con el código utilizado\footnote{Todo el contenido de este trabajo puede encontrarse en \href{https://github.com/NotNoe/TFG-Info}{este repositorio de github}.} y otro con la totalidad de librerías instaladas en el entorno de \emph{Python} sobre el que se ejecutó el código.

\chapter{Estado de la Cuestión}
\label{cap:estadoDeLaCuestion}

\begin{resumen}

\end{resumen}


\chapter{Cuestiones previas}
\label{cap:pre_datos}
Antes de empezar a entrenar modelos tenemos que decidir una serie de detalles que serán de gran importancia a lo largo del trabajo.

\section{Análisis de los datos}
El mayor problema de PTB-XL \citep{ptbxldb}, además de su tamaño relativamente pequeño, es el balanceo de clases, en la tabla \ref{tab:dist} podemos ver la distribución de las mismas. Esto puede causar varios problemas en el modelo, como por ejemplo:

\begin{table}[htbp] 
\centering
\begin{tabular}{|lllr|}
	\hline
	Número de registros & Superclase & Descripción & Porcentaje \\
	\hline
	9514 & NORM & ECG Normal & 43.64\% \\
	5469 & MI & Infarto de Miocardio & 25.08\% \\
	5235 & STTC & Cambio ST/T & 24.01\% \\
	4898 & CD & Transtorno de la conducción & 22.46\% \\
	2649 & HYP & Hipertrofia & 12.15\% \\
	\hline

\end{tabular}
\caption{Distribución de las superclases en PTB-XL}
\label{tab:dist}
\begin{tablenotes}
	\small
	\item La información de esta tabla ha sido extraída directamente del \href{https://physionet.org/content/ptb-xl/1.0.3/}{repositorio de PTB-XL}.
\end{tablenotes}
\end{table}

\begin{enumerate}
	\item \textbf{Sobreajuste hacia la clase mayoritaria:} Al haber bastantes más datos de entrenamiento de una clase (NORM) y menos de otra (HYP), el modelo puede aprender mejor los patrones que identifican las clases mayoritarias, haciendo que sepa distinguir peor las minoritarias, lo que en este caso podría reducir notablemente su capacidad de predicción de anomalías raras \citep{IData}.
	
	\item \textbf{Métricas no representativas:} Las métricas más comunes, como la exactitud, pueden ser poco informativas cuando hay un desbalance en los datos de prueba, ya que un modelo que predice siempre la clase mayoritaria puede tener una exactitud alta. Esto puede dificultar la evaluación real del rendimiento del modelo \citep{ClassOfIData}.
	
	\item \textbf{Dificultad en el entrenamiento:} Las redes neuronales profundas requieren de grandes cantidades de datos de entrenamiento para poder entender patrones complejos. Si una de las clases tiene muy pocos ejemplos, es muy probable que el modelo no sea capaz de predecirla correctamente \citep{Leevy}
\end{enumerate}

Para abordar estos problemas se podrían considerar varias estrategias, como hacer \emph{oversampling} o \emph{undersampling}. El \emph{oversampling} consiste en generar datos sintéticos a partir de los que ya tenemos para balancear las clases, pero esto no es una buena técnica cuándo los datos son complejos (como es el caso de un ECH), ya que no hay una técnica clara para crear datos sintéticos coherentes.

Por otro lado, el \emph{undersampling} hace que todas las clases se queden con el mismo número de candidatos que la clase minoritaria, lo que no es una técnica adecuada cuándo los datos de entrenamiento son reducidos desde un principio.

\section{Procesamiento de los datos}

Como es habitual en el campo de la inteligencia artificial, antes de poder utilizar unos datos hay que hacer cierto procesamiento para asegurarnos que son adecuados.

Lo primero que habría que hacer es quitar los datos repetidos o con valores inválidos, incompletos o corruptos, pero afortunadamente la base de datos que estamos utilizando ya ha sido revisada por sus creadores, por lo que podemos obviar este paso.

En procesamiento de señales (especialmente en señales que son muy sensibles a determinadas perturbaciones, como es el caso de los \ac{ECG}s) es muy importante aplicar determinados filtros antes de trabajar con las señales. En este trabajo utilizaremos los scripts que se utilizaron en el trabajo de \cite{TFGSergio} (que nos han sido facilitados por el autor). En concreto, los datos se pre-procesan de la siguiente manera:
\begin{itemize}
	\item Se normalizan todos para tener una frecuencia de 400Hz, que es con la que se entrenó al modelo original. Por tanto, tras hacer este procesamiento previo estaremos trabajando con vectores de 4096
	\item Se elimina el desplazamiento de la línea base. Como podemos ver en \cite{baseline}, es muy importante hacer esto antes de analizar un \ac{ECG}.
	\item Se elimina la interferencia de la línea de alimentación, lo que también es importante como podemos ver en \cite{powerline}
\end{itemize}

Por último, separamos los datos en tres conjuntos, siguiendo la división recomendada por la propia base de datos:
\begin{itemize}
	\item \textbf{Entrenamiento (\emph{train}):} El conjunto mayoritario (con un 80\% de los datos), que será usado para entrenar al modelo
	\item \textbf{Validación (\emph{validation}):} Este conjunto (que representa el 10\% de los datos) se utilizará para ajustar los parámetros del modelo en el entrenamiento del mismo.
	\item \textbf{Pruebas (\emph{test}):} Este conjunto (que está formado por el 10\% restante de los datos) es el que utilizaremos para obtener las diversas métricas de rendimiento del modelo.
\end{itemize}

\section{Métricas}
Antes de entrenar diversos modelos, tenemos que tener claro qué métricas estamos intentando maximizar, ya que no hay una métrica objetivamente mejor que las demás.

\subsection{Métricas habituales}
Entre las métricas más habituales podemos encontrar la \emph{F-$\beta$ Score}, \emph{precision} y \emph{recall}.

\subsubsection{Precision (Precisión)}
La precisión es la proporción de predicciones positivas que son realmente positivas, o más concretamente:
\[
\text{precision} = \frac{\text{Verdaderos positivos}}{\text{Verdaderos positivos + falsos positivos}}.
\]

Un valor alto de esta métrica indica que el modelo es bueno minimizando falsos positivos, es decir, cuándo el modelo predice que un dato no pertenece a una clase, esa predicción es fiable.

\subsubsection{Recall (Sensibilidad)}
La sensibilidad mide la proporción de casos positivos que el modelo predice correctamente, o más concretamente:
\[
\text{recall} = \frac{\text{Verdaderos positivos}}{\text{Verdaderos positivos + falsos negativos}}.
\]

Un valor alto de esta métrica indica que el modelo es bueno minimizando falsos positivos, es decir, cuándo el modelo predice que un dato pertenece a una clase, esa predicción es fiable.

\subsubsection{F-$\beta$ Score}
El F-$\beta$ Score es una media entre la precisión y el recall, la fórmula concreta es:
\[
F_\beta = (1+\beta^2)\times \frac{\text{precision} \times \text{recall}}{\beta^2\times\text{precision} + \text{recall}}.
\]
El valor más habitual para esto es $\beta=1$, que nos da la media armónica y permite valorar tanto la fiabilidad del modelo cuando predice positivo como negativo.

Esto es adecuado cuándo, por la naturaleza de un problema, el coste de los falsos positivos es similar al de los falsos negativos, pero no es nuestro caso. En modelos aplicados a la salud, es mucho más importante predecir las anomalías correctamente (ya que de esto puede depende la salud de una persona) que predecir correctamente la ausencia de anomalías.

Los valores de $\beta=0.5,2$ hacen que tenga más peso la precisión y el recall respectivamente, por lo que la primera es más adecuada para cuándo los falsos positivos tienen un coste muy alto y la segunda para cuándo son los falsos negativos los que tienen el coste más alto.

\subsection{Cálculo por clase vs. Promedio binario}
Todas las métricas que hemos listado anteriormente están definidas para clasificadores binarios, pero nuestro clasificador es multietiqueta, por lo que es necesario adaptarlas. En este trabajo consideraremos dos enfoques, el cálculo por clase y el promedio binario.

\subsubsection{Cálculo por clase}
Este es el enfoque más sencillo de todos. Consideramos nuestro clasificador multietiqueta como uno binario para cada una de sus etiquetas, y calculamos las métricas para cada una de las clases.

Este enfoque permite ver el desempeño del modelo en cada una de sus clases, lo que permite entender mejor cuáles son sus debilidades y fortalezas. El principal problema que presenta este método es que no da un único valor para comparar modelos, por lo que puede ser difícil determinar qué modelo es el óptimo.

\subsubsection{Promedio binario}
Este enfoque (también conocido como \emph{one-vs-rest}) consiste en hacer el promedio de las métricas obtenidas en el enfoque anterior para cada clase.

\subsection{Métricas para nuestro problema}
Tras realizar el análisis de las posibles métricas que implementar, hemos decidido calcular y mostrar varias métricas para cada modelo, y elegir una que consideramos mejor para afirmar qué modelo es el mejor. Las métricas que mostraremos son las siguientes:

\begin{itemize}
	\item Para cada una de las clases:
	\begin{itemize}
		\item Precisión.
		\item Recall.
		\item F-1 Score.
	\end{itemize}
	\item Precisión global calculada como promedio binario.
	\item Recall global calculado como promedio binario.
	\item F-1 Score global calculado como promedio binario.
	\item F Score ajustada, una métrica personalizada que definiremos a continuación.
\end{itemize}

Todas las métricas que mostraremos, salvo la personalizada, tienen el objetivo de entender mejor cómo funciona el modelo, pero no de compararlo. La F Score ajustada será la que utilizaremos para determinar qué modelo consideramos óptimo.

En nuestro problema tenemos cinco etiquetas. Una de ellas representa un \ac{ECH} normal, mientras que las demás representan diversas anomalías. Dado que nuestro objetivo es que el modelo identifique lo mejor posible las anomalías (cuándo las haya), el coste de los falsos negativos en las etiquetas de anomalías es muy alto, mientras que el coste de los falsos positivos en la etiqueta normal es muy alto.

Por ello, definiremos la F Score ajustada como la media de la F-0.5 Score de la clase normal y las  F-2 Score del resto de etiquetas. Esto nos permite tener una métrica que tiene en cuenta tanto reducir falsos negativos como falsos positivos en todas las etiquetas, pero dando más peso a los falsos negativos o positivos dependiendo de la etiqueta concreta.

La fórmula concreta de la métrica sería:
\begin{equation*}
	\text{F Score ajustada} = \frac{F-0.5(\text{NORM}) + \sum_{i \neq \text{NORM}}F-2(i)}{\text{Número de clases}}
\end{equation*}


\section{Transformaciones}
\chapter{Entrenamiento y resultados}
\label{cap:train}
\begin{resumen}
	Este capítulo describe el proceso de entrenamiento de los modelos, detalla la configuración utilizada y presenta las métricas cuantitativas que permiten comparar su desempeño. Posteriormente se discuten los resultados obtenidos.
\end{resumen}

\section{Modelos modificados entrenados}
Modificamos la capa de entrada del \emph{gold standard} para poder entrenar modelos con datos en matrices bidimensionales de cualquier tamaño, lo que nos permitió (con una modificación muy pequeña del \emph{gold standard}) entrenar modelos con imágenes como datos de entrenamiento.

Al transformar cada una de las doce derivaciones obtenemos una matriz (de dimensiones dependientes exclusivamente del tamaño de la señal de entrada y de los parámetros de la transformada). Esto significa que al aplicar la misma transformada a las doce derivaciones de un ECG, obtenemos doce matrices de las mismas dimensiones, pero el modelo necesita solamente una matriz. Para abordar este problema, concatenamos las matrices en el eje de su altura, lo que nos permite utilizar el modelo modificado con los datos de las transformadas.

En concreto, los modelos modificados que entrenamos son los siguientes:
\begin{itemize}
	\item \textbf{Modelo STFT}: Entrenado con los datos transformados utilizando los parámetros especificados en la subsección \ref{subsec:stft}
	\item \textbf{Modelo CWT Ricker}: Entrenado con los datos transformados utilizando los parámetros especificados en la subsección \ref{subsec:cwt} con el \emph{wavelet} de Ricker.
	\item \textbf{Modelo CWT Morlet}: Entrenado con los datos transformados utilizando los parámetros especificados en la subsección \ref{subsec:cwt} con el \emph{wavelet} de Morlet.
\end{itemize}

\section{Configuración del entrenamiento}
\subsection{\emph{Hardware} y \emph{software}}
Todos los entrenamientos se han realizado en \emph{bujaruelo}, una máquina del departamento de arquitectura de computadores y automática de la facultad de informática de la universidad.
\subsubsection{\emph{Hardware}}
\begin{itemize}
	\item \textbf{Procesador}: Intel(R) Xeon(R) CPU E5-2695 v3 @2.3GHz
	\item  \textbf{GPUs}: El equipo tiene en total cuatro gráficas, dos de cada uno de los modelos siguientes:
	\begin{itemize}
		\item NVIDIA GeForce GTX 1080
		\item NVIDIA GeForce GTX 980
	\end{itemize}
	\item \textbf{Discos}: El equipo cuenta con dos discos de 1TB, además del disco donde está montado el sistema operativo, de 74.5GB.
\end{itemize}
\subsubsection{\emph{Software}}
En el Anexo \ref{anexo:librerias} puede encontrarse una tabla con la totalidad de las librerías instaladas en el entorno de python que hemos utilizado (así como sus respectivas versiones), pero las más destacables para el trabajo son las siguientes:

\begin{itemize}
	\item \textbf{Python}: Utilizamos la versión 3.11.7
	\item \textbf{TensorFlow}: Utilizamos la versión 2.15, por motivos de compatibilidad con el modelo de Ribeiro.
	\item \textbf{CUDA y cuDNN}: Estas librerías son las que utiliza TensorFlow para realizar los entrenamientos en las gráficas de NVIDIA. En nuestro caso utilizamos las versiones 12.2 y 8.9 respectivamente.
\end{itemize}

Además, utilizamos \textbf{Matplotlib} 3.8.2 y \textbf{Seaborn} 0.13.2 para generar los gráficos y \textbf{Pandas} 2.2.0 y \textbf{Scikit-learn} 1.3.0 para procesar y analizar los datos y generar métricas.


\subsection{Parámetros de entrenamiento}
Para entrenar a los modelos se ha utilizado el script de entrenamiento que aparece en el repositorio del \emph{gold standard}, ligeramente modificado para que trabaje con los datos transformados, que tienen dimensiones diferentes a los datos originales.

\section{Resultados cuantitativos}
En la tabla \ref{tab:resultados_modelos} podemos ver los resultados de todos los entrenamientos.

\begin{table}[htbp]
	\centering
	\resizebox{\textwidth}{!}{%
	\begin{tabular}{|l|l|ccccc|c|}
		\toprule
		\textbf{Modelo} & \textbf{Métrica} & \textbf{NORM} & \textbf{MI} & \textbf{STTC} & \textbf{CD} & \textbf{HYP} & \textbf{Global} \\
		\midrule
		% Modelo Original
		\multirow{4}{*}{\emph{Gold Standard}} 
		& F-1 Score         & 0.835 & 0.680 & 0.718 & 0.705 & 0.413 & 0.737 \\
		& Recall            & 0.849 & 0.603 & 0.665 & 0.647 & 0.284 & 0.687 \\
		& Precisión         & 0.822 & 0.778 & 0.780 & 0.773 & 0.759 & 0.796 \\
		& F Score ajustada  & --    & --    & --    & --    & --    & 0.628 \\
		\midrule
		% Modelo STFT
		\multirow{4}{*}{STFT} 
		& F-1 Score         & 0.826 & 0.536 & 0.687 & 0.672 & 0.466 & 0.706 \\
		& Recall            & 0.853 & 0.433 & 0.640 & 0.589 & 0.338 & 0.650 \\
		& Precisión         & 0.801 & 0.701 & 0.741 & 0.783 & 0.750 & 0.772 \\
		& F Score ajustada  & --    & --    & --    & --    & --    & 0.587 \\
		\midrule
		% Modelo CWT Morlet
		\multirow{4}{*}{CWT Morlet} 
		& F-1 Score         & 0.817 & 0.457 & 0.645 & 0.630 & 0.395 & 0.644 \\
		& Recall            & 0.857 & 0.342 & 0.636 & 0.548 & 0.270 & 0.623 \\
		& Precisión         & 0.780 & 0.688 & 0.654 & 0.741 & 0.732 & 0.734 \\
		& F Score ajustada  & --    & --    & --    & --    & --    & 0.540 \\
		\midrule
		% Modelo CWT Ricker
		\multirow{4}{*}{CWT Ricker} 
		& F-1 Score         & 0.775    & 0.316    & 0.629    & 0.587    & 0.426    & 0.628    \\
		& Recall            & 0.774    & 0.224    & 0.633    & 0.520    & 0.319    & 0.572    \\
		& Precisión         & 0.776    & 0.538    & 0.626    & 0.675    & 0.639    & 0.696    \\
		& F Score ajustada  & --    & --    & --   & --    & --    & 0.512    \\
		\bottomrule
	\end{tabular}
	}
	\caption{Resultados de los modelos (con tres dígitos decimales de precisión)}
	\label{tab:resultados_modelos}
\end{table}

\section{Análisis de los resultados}
Los datos de la sección anterior muestran que al entrenar el \emph{gold standard} con diversas transformadas arroja unos resultados destacablemente peores en casi todas las métricas. Esto puede deberse a:
\begin{itemize}
	\item Una curva de aprendizaje más compleja para los modelos que utilizan transformadas, ya que generalizar patrones en estas podría ser más complejo para la red neuronal que hacerlo sobre las señales sin transformar. Esta hipótesis además se apoya en el hecho de que los modelos transformados hayan necesitado varios días para entrenarse, a diferencia del \emph{gold standard} (que se pudo entrenar en tan solo unas horas).
	\item Una elección incorrecta de los parámetros para las transformadas, ya que, si bien estos se han elegido con el objetivo de cubrir los rangos de frecuencias donde aparecen las anomalías que estamos buscando, podrían elegirse una cantidad inmensa de configuraciones diferentes.
	\item Limitaciones de la arquitectura del \emph{gold standard}, ya que utiliza redes Conv1D, que se especializan en detectar patrones en series temporales en lugar de en imágenes.
	\item Al transformar el modelo en un clasificador binario, se ha utilizado un \emph{threshold} de 0.5 para decidir si el ECG entra dentro de cada una de las etiquetas. Es posible que utilizando diferentes valores se obtengan mejores predicciones.
\end{itemize}

\chapter{Explicabilidad y validación clínica}
\label{cap:explicabilidad}
\begin{resumen}
	En este capítulo describimos cómo se generan y utilizan los \emph{saliency maps} para explicar las decisiones del \emph{gold standard}, detallamos la validación preliminar con un especialista y exponemos las mejoras introducidas a partir de su retroalimentación.
\end{resumen}

\section{Método de explicabilidad utilizado}
La única técnica de explicabilidad implementada en este trabajo son los \emph{saliency maps} basados en gradientes, aplicados al \emph{gold standard} exclusivamente. Esta decisión se debe a las siguientes razones:

\begin{enumerate}
	\item \textbf{Desempeño superior del \emph{gold standard}}: En el Capítulo \ref{cap:train} hemos visto que el \emph{gold standard} obtiene mejores resultados en casi todas las métricas evaluadas.
	\item \textbf{Imposibilidad de Grad-CAM en modelos transformados}: Como las redes transformadas mantienen la arquitectura con capas Conv1D, no se pueden aplicar métodos como Grad-CAM a la representación bidimensional. Además, los \emph{saliency maps} no ofrecen una explicación clara sobre imágenes, ya que tratan cada una de las filas de la entrada como datos independientes.
\end{enumerate}
\section{Resultados de explicabilidad}
Aplicando la librería de explicabilidad TSInterpret\footnote{Puede encontrarse su documentación en el siguiente \href{https://fzi-forschungszentrum-informatik.github.io/TSInterpret/}{enlace}.}, hemos explicado cinco casos del conjunto de test para cada clase que cumplen las siguientes condiciones:
\begin{itemize}
	\item El valor real de la etiqueta es solamente una condición, es decir, el valor esperado es 1.0 para una de las clases y 0.0 para el resto.
	\item La predicción es perfecta, es decir, la única etiqueta en la que el modelo clasifica el ECG es la que tiene un valor real de 1.0.
	\item La clase HYP no tiene ningún caso así, debido a que es la clase minoritaria, por lo que no hemos podido aplicar explicabilidad a esta clase.
\end{itemize}
Todas las explicaciones realizadas pueden encontrarse en la siguiente \href{https://drive.google.com/drive/folders/1EOW2RsL3Ub2UQ1NiEeRpMVd9SsnQpBkf?usp=drive_link}{carpeta}, pero en la Figura \ref{fig:explicaciones} podemos ver una imagen con la explicación de la primera derivación de un ECG de cada clase (excepto HYP).

\begin{figure}
	\centering
	\includegraphics[width=0.9\linewidth]{Imagenes/Vectorial/explanations/NORM.png}
	\includegraphics[width=0.9\linewidth]{Imagenes/Vectorial/explanations/MI.png}
	\includegraphics[width=0.9\linewidth]{Imagenes/Vectorial/explanations/STTC.png}
	\includegraphics[width=0.9\linewidth]{Imagenes/Vectorial/explanations/CD.png}
	\caption{Explicaciones de la primera derivación de un ECG de las clases NORM, MI, STTC, CD (respectivamente)}.
	\label{fig:explicaciones}.
\end{figure}

En la Figura \ref{fig:explicacionvs} puede verse una comparación entre nuestra explicación de un infarto de miocardio y la de otro artículo, y se puede comprobar que, en efecto, los modelos se fijan en las mismas zonas (en este caso concreto, después del valle más pronunciado).

\begin{figure}
	\centering
	\includegraphics[width=0.9\linewidth]{Imagenes/Vectorial/Feature_7.png}
	\includegraphics[width=0.9\linewidth]{Imagenes/Vectorial/V2.png}
	\caption{Explicación de un infarto de miocardio en nuestro modelo (arriba) y en un artículo de investigación (abajo). Fuente: \cite{gustafsson2022deep}.}
	\label{fig:explicacionvs}
\end{figure}

\section{Validación con los expertos médicos}
Tras hacer las explicaciones, se las mostramos a varios médicos para obtener feedback de éstas. Su opinión general es que es una herramienta con potencial de ser muy útil, y que generalmente detecta correctamente zonas donde puede verse la anomalías, pero hizo los siguientes comentarios de cara a mejorarlo:
\begin{enumerate}
	\item \textbf{Heterogeneidad de la explicación} \\
	El experto señaló que, salvo excepciones, las anomalías cardíacas se presentan en todos los latidos, no solo en algunos, por lo que es extraño que la explicación señale únicamente las anomalías en algunos latidos.
	\item \textbf{Alternativa de presentación} \\
	El experto sugirió reducir las imágenes y mostrar únicamente dos latidos del corazón, ya que esto es una forma habitual de visualizar y explicar anomalías a los médicos en formación.
	\item \textbf{Conformidad con la práctica clínica habitual} \\
	El experto señaló que la presentación que hemos hecho del ECG se separa bastante de la presentación habitual de estos. Indica que si las imágenes estuvieran estandarizadas según la práctica clínica habitual, podría ser una herramienta muy útil para la enseñanza, pero que en su estado actual podría llevar a confusión a los alumnos.
\end{enumerate}

En general el experto tiene la sensación de que esta herramienta tiene el potencial de ser algo muy útil, principalmente en el ámbito de la enseñanza, pero que necesita implementar algunos cambios antes de ser utilizada.

\section{Modificaciones a partir del feedback}
Para poder hacer la explicación homogénea en cada latido o reducir la imagen a dos latidos representativos, necesitaríamos detectar donde empieza y acaba cada latido, así como las distintas partes de la señal. Si bien es cierto que esto es algo posible, se escapa del alcance de este trabajo, por lo que no lo implementaremos.

Respecto a la representación habitual del ECG, el motivo de que esté presentado de esta forma es que hemos utilizado la propia librería de explicabilidad para dibujar las gráficas, y ésta trabaja con series temporales en general. Para abordar este problema y mejorar la presentación, hemos modificado la imagen que genera librería \emph{ecg\_plot}, de manera que también pueda plasmar los colores relativos a la explicación por encima. En la Figura \ref{fig:explicacion_modificada} podemos ver una explicación mejorada para cada clase (excepto HYP).

\begin{figure}
	\centering
	\includegraphics[width=0.9\textwidth]{Imagenes/Vectorial/better_explanations/NORM.png}
	\includegraphics[width=0.9\textwidth]{Imagenes/Vectorial/better_explanations/MI.png}
	\includegraphics[width=0.9\textwidth]{Imagenes/Vectorial/better_explanations/STTC.png}
	\includegraphics[width=0.9\textwidth]{Imagenes/Vectorial/better_explanations/CD.png}
	\caption{Explicación de un ECG de cada clase (excepto HYP) pintada sobre la librería \emph{ecg\_plot}}.
	\label{fig:explicacion_modificada}
\end{figure}
\chapter{Conclusiones y Trabajo Futuro}
\label{cap:conclusiones}

\section{Conclusiones}
A lo largo de este trabajo hemos demostrado que el modelo de Ribeiro funciona de manera adecuada aún entrenado con una base de datos más pequeña, lo que señala que es una arquitectura bastante sólida y prometedora.

No obstante, a la hora de introducir transformaciones, hemos descubierto que la arquitectura del modelo de Ribeiro no es adecuada para manejar imágenes, como es el caso de las transformadas. Esto no significa que el modelo de Ribeiro no pueda ser adaptado para trabajar con imágenes de manera adecuada, ya que una arquitectura con unos resultados tan buenos es muy prometedora.

Es importante señalar que el campo del \emph{deep learning} aplicado a la medicina es relativamente nuevo, y se publican trabajos innovadores y avances muy frecuentemente, por lo que el uso de transformadas para el análisis de ECGs sigue siendo una posible dirección de avance.

En cuanto a la explicabilidad, los \emph{saliency maps} basados en gradientes han mostrado ser un recurso prometedor para entender qué partes de la señal del ECG resultan relevantes para la red. Refinar estas explicaciones podría convertirlas en una herramienta de aprendizaje y educación útil tanto para estudiantes como para profesionales sanitarios que busquen familiarizarse con la IA. Además hemos visto como los profesionales reaccionaron muy positivamente a la existencia de herramientas que permitan explicar las decisiones de los modelos, y están convencidos de que, una vez se mejoren, van a ser un recurso muy importante en la educación.

\section{Trabajo futuro}
Dado el potencial del modelo y las limitaciones detectadas, se plantean varias líneas de trabajo para mejorar y ampliar los resultados:
\begin{enumerate}
	\item \textbf{Modificar la arquitectura para admitir capas Conv2D} \\
	Esta adaptación, si bien requeriría de un trabajo significativo, permitiría explotar mejor las ventajas de las transformaciones de la señal.
	\item \textbf{Explorar transformadas y parámetros adicionales} \\
	Siguiendo la línea anterior, una vez obtenida una arquitectura más adecuada para imágenes, podrían probarse otros parámetros a la hora de hacer las transformadas y estudiar su impacto.
	\item \textbf{Especializar el modelo en una anomalía} \\
	Transformar el modelo en un clasificador binario permitiría elegir los parámetros de las transformadas de manera más precisa, ya que distintas anomalías se manifiestan en distintas frecuencias.
	\item \textbf{Cambiar los umbrales de detección} \\
	En el trabajo de Ribeiro, en lugar de utilizar el umbral de 0.5 para clasificar las etiquetas, se calcularon los umbrales que optimizaban los resultados sobre el conjunto de validación. Esto es algo que, si bien en este trabajo no hemos abordado, podría mejorar los resultados de todos los modelos que hemos entrenado.
	\item \textbf{Delineación de señales en la explicación} \\
	Como señaló el experto, mediante delineación de señales podrían reducirse las explicaciones a dos latidos y asegurarse de que estas son homogéneas en ambos latidos. Estas mejoras harían viable el uso de estas explicaciones en el ámbito docente.
\end{enumerate}




\begin{otherlanguage}{english}
\chapter*{Introduction}
\label{cap:introduction}
\addcontentsline{toc}{chapter}{Introduction}
\section{Motivation}
AI has transformed numerous fields. In particular, the medical field has experienced significant advances through the application of artificial intelligence to diagnostics and data analysis. One of the most promising areas is the automated interpretation of signals in an ECG, which is essential for the early detection of heart diseases. Cardiovascular diseases are one of the leading causes of death globally \citep{whocvd}, highlighting the need for fast and accurate diagnostic methods.

The analysis of ECG's has historically been performed by healthcare professionals, such as cardiologists, due to the complexity and variability of the signals. However, this process is slow, costly, and highly dependent on the specialist’s experience. By adopting \emph{Deep Learning} models, such as convolutional neural networks (CNN), the goal is to automate and enhance this analysis, providing rapid results that, in many cases, are comparable to those obtained by human experts \citep{hannun_cardiologist_2019}. This approach not only promises to increase diagnostic efficiency but also improve accuracy and reduce the workload of medical staff.

Nevertheless, for these tools to gain trust in clinical settings and meet quality and ethical standards, they must be explainable \citep{Molnar2019}. The explainability of AI models permits understanding and justifying the decisions made during the analysis of cardiac signals, a critical factor in high-impact applications such as medicine, where an error can directly affect patient health \citep{Goodman2017}.

Throughout this work, we will explore various neural network architectures and signal transformation techniques to classify and detect abnormalities in ECGs, aiming to develop and refine an ECG classifier model for four groups of cardiac anomalies. In addition, we will incorporate explainability methods so that model results can be interpreted and validated by healthcare professionals, which is essential for clinical adoption of these technologies.

\section{Objectives}
The main objective of this work is to modify, train, and evaluate the model originally proposed by \cite{ribeiro} (hereinafter, the “original model”) by applying it to the public PTB-XL database \citep{ptbxldb}. To achieve this, we will introduce a modification in the model’s input layer that allows the inclusion of transforms (e.g., STFT or CWT) to investigate whether converting the signal into different representations can improve performance. However, no changes are made to the model’s internal architecture.

Based on the above, the specific objectives are:

\begin{enumerate}
	\item \textbf{Train the original model} using the PTB-XL database.
	\item \textbf{Modify the original model} so that it accepts signal transformations, permitting the training of alternative versions of the model with transformed data.
	\item \textbf{Compare the performance} of each variant with the original model using metrics such as F1-Score, analyzing whether the transformations are beneficial.
	\item \textbf{Apply an explainability method} so that a specialist can understand which parts of the ECG signal influence the prediction. For reasons to be discussed later, this will only be applied to the original model.
\end{enumerate}

\section{Document structure}
This document is organized into six chapters, whose contents are described below:
\begin{itemize}
	\item \textbf{Chapter 2: Estado del arte} \\
	Basic concepts related to electrocardiograms are presented, highlighting the importance of their waves (P, QRS, T) and the most common anomalies typically detected. Furthermore, the literature related to the use of neural networks in the analysis of ECG's is reviewed, and relevant databases (such as PTB-XL) are described.
	
	\item  \textbf{Chapter 3: Metodología y preparación de datos} \\
	The process for handling ECG's is detailed, including the division into training, validation, and test sets. In addition, the metrics used to evaluate the models and the transformations employed are described.
	
	\item \textbf{Chapter 4: Entrenamiento y resultados}\\
	The training conditions for each model (including modifications to the original model) and the libraries used are explained, and the quantitative results obtained through the metrics are presented and compared.
	
	\item \textbf{Chapter 5: Explicabilidad}\\
	The gradient-based \emph{saliency maps} explainability method applied to Ribeiro’s model is described. In addition, the perspective of a specialist physician analyzing the generated explanations is included, and reflections on the method and potential improvements are discussed.
	
	\item \textbf{Chapter 6: Conclusiones y trabajo futuro}\\
	The conclusions derived from the results are integrated, assessing the extent to which the stated objectives have been met. Furthermore, the main contributions of this work are identified, and future research directions are proposed (such as the inclusion of Conv2D layers).
\end{itemize}

Finally, a bibliography section is included, containing all the sources consulted throughout the document, as well as an appendix with the code used\footnote{All the content of this work can be found in \href{https://github.com/NotNoe/TFG-Info}{this GitHub repository}.} and another with the physician’s comments transcribed.



\chapter*{Conclusions and Future Work}
\label{cap:conclusions}
\addcontentsline{toc}{chapter}{Conclusions and Future Work}
\section{Conclusions}
Over the course of this work, the applicability of Ribeiro’s model for classifying ECG's from the PTB-XL database has been demonstrated, achieving acceptable results when processing the original signal. However, when transformations (STFT, CWT, etc.) are introduced and the same one-dimensional convolution-based architecture is maintained, the results deteriorate significantly. This observation suggests that Ribeiro’s architecture, in its current form, is not suitable for handling images, thus requiring substantial modifications to fully exploit the time-frequency representations offered by the transforms.

Regarding explainability, gradient-based \emph{saliency maps} have proven to be a promising resource for understanding which parts of the ECG signal are relevant to the network. Refining these explanations could turn them into a valuable learning and educational tool for both students and healthcare professionals seeking to familiarize themselves with AI.

\section{Future Work}
Given the model’s potential and the identified limitations, various lines of work are proposed to improve and extend the results:
\begin{enumerate}
	\item \textbf{Modify the architecture to support Conv2D layers} \\
	Although this adaptation would require significant effort, it would allow for better exploitation of the advantages of signal transformations.
	\item \textbf{Explore additional transforms and parameters} \\
	Following the previous line, once an architecture more suitable for images is obtained, other parameters could be tested when performing the transforms, and their impact could be studied.
	\item \textbf{Specialize the model in one anomaly} \\
	Transforming the model into a binary classifier would allow for more precise parameter selection for the transforms, since different anomalies manifest in different frequency ranges.
	\item \textbf{Signal delineation in explanations} \\
	As noted by the expert, through signal delineation it would be possible to reduce explanations to two heartbeats and ensure they are homogeneous in both. These improvements would make the use of these explanations viable in educational settings.
\end{enumerate}

\end{otherlanguage}

%\include{Capitulos/ContribucionesPersonales}

%
% Bibliografía
%
% Si el TFM se escribe en inglés, editar TeXiS/TeXiS_bib para cambiar el
% estilo de las referencias
\include{Cascaras/bibliografia}


% Apéndices
\appendix
\chapter{Código}
\label{anexo:codigo}
\lstinputlisting[language=Python,caption={Código empleado para pintar las explicaciones sobre la imagen generada por \emph{ecg\_plot}.},lastline=86]{../Codigo/scripts/explain_beautify.py}

\lstinputlisting[language=Python,caption={Código empleado para buscar los ECGs adecuados para explicaciones.}]{../Codigo/scripts/search\_perfect.py}

\lstinputlisting[language=Python,caption={Clase creada para hacer las transformaciones de los datos.}]{../Codigo/scripts/Transformaciones.py}

\lstinputlisting[language=Python,caption={Clase creada para generar métricas y guardarlas en un json.}]{../Codigo/scripts/Metrics.py}
\chapter{Librerías utilizadas}
\label{anexo:librerias}

\begin{longtable}{|l|l|}
	\hline
	\textbf{Librería} & \textbf{Versión} \\ \hline
	\hline
	\endfirsthead
	\hline
	\textbf{Librería} & \textbf{Versión} \\ \hline
	\endhead
	\hline
	\endfoot
	\hline
	\endlastfoot
	\_libgcc\_mutex      & 0.1=main        \\ \hline
	\_openmp\_mutex      & 5.1=1\_gnu      \\ \hline
	bzip2                & 1.0.8=h7b6447c\_0 \\ \hline
	ca-certificates      & 2023.12.12=h06a4308\_0 \\ \hline
	ld\_impl\_linux-64   & 2.38=h1181459\_1 \\ \hline
	libffi               & 3.4.4=h6a678d5\_0 \\ \hline
	libgcc-ng            & 11.2.0=h1234567\_1 \\ \hline
	libgomp              & 11.2.0=h1234567\_1 \\ \hline
	libstdcxx-ng         & 11.2.0=h1234567\_1 \\ \hline
	libuuid              & 1.41.5=h5eee18b\_0 \\ \hline
	ncurses              & 6.4=h6a678d5\_0 \\ \hline
	openssl              & 3.0.13=h7f8727e\_0 \\ \hline
	pip                  & 23.3.1=py311h06a4308\_0 \\ \hline
	python               & 3.11.7=h955ad1f\_0 \\ \hline
	readline             & 8.2=h5eee18b\_0 \\ \hline
	setuptools           & 68.2.2=py311h06a4308\_0 \\ \hline
	sqlite               & 3.41.2=h5eee18b\_0 \\ \hline
	tk                   & 8.6.12=h1ccaba5\_0 \\ \hline
	wheel                & 0.41.2=py311h06a4308\_0 \\ \hline
	xz                   & 5.4.5=h5eee18b\_0 \\ \hline
	zlib                 & 1.2.13=h5eee18b\_0 \\ \hline
	absl-py              & 2.1.0 \\ \hline
	aiohttp              & 3.9.3 \\ \hline
	aiosignal            & 1.3.1 \\ \hline
	asttokens            & 2.4.1 \\ \hline
	astunparse           & 1.6.3 \\ \hline
	attrs                & 23.2.0 \\ \hline
	autograd             & 1.7.0 \\ \hline
	beautifulsoup4       & 4.12.3 \\ \hline
	bleach               & 6.1.0 \\ \hline
	cachetools           & 5.3.2 \\ \hline
	captum               & 0.7.0 \\ \hline
	certifi              & 2024.2.2 \\ \hline
	cffi                 & 1.16.0 \\ \hline
	charset-normalizer   & 3.3.2 \\ \hline
	cloudpickle          & 3.1.0 \\ \hline
	comm                 & 0.2.1 \\ \hline
	contourpy            & 1.2.0 \\ \hline
	cycler               & 0.12.1 \\ \hline
	deap                 & 1.4.1 \\ \hline
	debugpy              & 1.8.0 \\ \hline
	decorator            & 5.1.1 \\ \hline
	defusedxml           & 0.7.1 \\ \hline
	deprecated           & 1.2.13 \\ \hline
	ecg-plot             & 0.2.8 \\ \hline
	et-xmlfile           & 1.1.0 \\ \hline
	executing            & 2.0.1 \\ \hline
	fastjsonschema       & 2.19.1 \\ \hline
	filelock             & 3.13.4 \\ \hline
	flatbuffers          & 23.5.26 \\ \hline
	fonttools            & 4.48.1 \\ \hline
	frozenlist           & 1.4.1 \\ \hline
	fsspec               & 2024.3.1 \\ \hline
	gast                 & 0.5.4 \\ \hline
	google-auth          & 2.27.0 \\ \hline
	google-auth-oauthlib & 1.2.0 \\ \hline
	google-pasta         & 0.2.0 \\ \hline
	grpcio               & 1.60.1 \\ \hline
	h5py                 & 3.10.0 \\ \hline
	idna                 & 3.6 \\ \hline
	ipykernel            & 6.29.2 \\ \hline
	ipython              & 8.21.0 \\ \hline
	jedi                 & 0.19.1 \\ \hline
	jinja2               & 3.1.3 \\ \hline
	joblib               & 1.3.2 \\ \hline
	jsonschema           & 4.21.1 \\ \hline
	jsonschema-specifications & 2023.12.1 \\ \hline
	jupyter-client       & 8.6.0 \\ \hline
	jupyter-core         & 5.7.1 \\ \hline
	jupyterlab-pygments  & 0.3.0 \\ \hline
	keras                & 2.15.0 \\ \hline
	kiwisolver           & 1.4.5 \\ \hline
	libclang             & 16.0.6 \\ \hline
	lightning-utilities  & 0.11.2 \\ \hline
	llvmlite             & 0.43.0 \\ \hline
	locket               & 1.0.0 \\ \hline
	markdown             & 3.3.4 \\ \hline
	markupsafe           & 2.1.5 \\ \hline
	matplotlib           & 3.8.2 \\ \hline
	matplotlib-inline    & 0.1.6 \\ \hline
	mistune              & 3.0.2 \\ \hline
	ml-dtypes            & 0.2.0 \\ \hline
	mpmath               & 1.3.0 \\ \hline
	multidict            & 6.0.5 \\ \hline
	nbclient             & 0.9.0 \\ \hline
	nbconvert            & 7.16.1 \\ \hline
	nbformat             & 5.9.2 \\ \hline
	nest-asyncio         & 1.6.0 \\ \hline
	networkx             & 3.3 \\ \hline
	numba                & 0.60.0 \\ \hline
	numpy                & 1.26.4 \\ \hline
	nvidia-cublas-cu12   & 12.1.3.1 \\ \hline
	nvidia-cuda-cupti-cu12 & 12.1.105 \\ \hline
	nvidia-cuda-nvcc-cu12 & 12.2.140 \\ \hline
	nvidia-cuda-nvrtc-cu12 & 12.1.105 \\ \hline
	nvidia-cuda-runtime-cu12 & 12.1.105 \\ \hline
	nvidia-cudnn-cu12    & 8.9.2.26 \\ \hline
	nvidia-cufft-cu12    & 11.0.2.54 \\ \hline
	nvidia-curand-cu12   & 10.3.2.106 \\ \hline
	nvidia-cusolver-cu12 & 11.4.5.107 \\ \hline
	nvidia-cusparse-cu12 & 12.1.0.106 \\ \hline
	nvidia-nccl-cu12     & 2.19.3 \\ \hline
	nvidia-nvjitlink-cu12 & 12.2.140 \\ \hline
	nvidia-nvtx-cu12     & 12.1.105 \\ \hline
	oauthlib             & 3.2.2 \\ \hline
	opencv-python        & 4.6.0.66 \\ \hline
	openpyxl             & 3.1.2 \\ \hline
	opt-einsum           & 3.3.0 \\ \hline
	packaging            & 23.2 \\ \hline
	pandas               & 2.2.0 \\ \hline
	pandocfilters        & 1.5.1 \\ \hline
	parso                & 0.8.3 \\ \hline
	partd                & 1.2.0 \\ \hline
	patsy                & 1.0.1 \\ \hline
	pexpect              & 4.9.0 \\ \hline
	pillow               & 10.2.0 \\ \hline
	platformdirs         & 4.2.0 \\ \hline
	prompt-toolkit       & 3.0.43 \\ \hline
	protobuf             & 4.23.4 \\ \hline
	psutil               & 5.9.8 \\ \hline
	ptyprocess           & 0.7.0 \\ \hline
	pure-eval            & 0.2.2 \\ \hline
	pyaml                & 24.9.0 \\ \hline
	pyarrow              & 15.0.2 \\ \hline
	pyasn1               & 0.5.1 \\ \hline
	pyasn1-modules       & 0.3.0 \\ \hline
	pycparser            & 2.21 \\ \hline
	pygments             & 2.17.2 \\ \hline
	pymop                & 0.2.4 \\ \hline
	pyparsing            & 3.1.1 \\ \hline
	python-dateutil      & 2.8.2 \\ \hline
	pytorch-lightning    & 2.2.1 \\ \hline
	pyts                 & 0.13.0 \\ \hline
	pytz                 & 2024.1 \\ \hline
	pyyaml               & 6.0.1 \\ \hline
	pyzmq                & 25.1.2 \\ \hline
	referencing          & 0.33.0 \\ \hline
	requests             & 2.31.0 \\ \hline
	requests-oauthlib    & 1.3.1 \\ \hline
	rpds-py              & 0.18.0 \\ \hline
	rsa                  & 4.9 \\ \hline
	scikit-base          & 0.11.0 \\ \hline
	scikit-learn         & 1.3.0 \\ \hline
	scikit-optimize      & 0.10.2 \\ \hline
	scipy                & 1.14.1 \\ \hline
	seaborn              & 0.13.2 \\ \hline
	shap                 & 0.46.0 \\ \hline
	six                  & 1.16.0 \\ \hline
	sktime               & 0.34.1 \\ \hline
	slicer               & 0.0.8 \\ \hline
	soundfile            & 0.12.1 \\ \hline
	soupsieve            & 2.5 \\ \hline
	stack-data           & 0.6.3 \\ \hline
	statsmodels          & 0.14.4 \\ \hline
	stumpy               & 1.13.0 \\ \hline
	sympy                & 1.12 \\ \hline
	tensorboard          & 2.15.1 \\ \hline
	tensorboard-data-server & 0.7.2 \\ \hline
	tensorflow           & 2.15.0.post1 \\ \hline
	tensorflow-estimator & 2.15.0 \\ \hline
	tensorflow-io-gcs-filesystem & 0.36.0 \\ \hline
	termcolor            & 2.4.0 \\ \hline
	tf-explain           & 0.3.1 \\ \hline
	threadpoolctl        & 3.2.0 \\ \hline
	tinycss2             & 1.2.1 \\ \hline
	toolz                & 1.0.0 \\ \hline
	torch                & 2.2.2 \\ \hline
	torchcam             & 0.4.0 \\ \hline
	torchmetrics         & 1.3.2 \\ \hline
	tornado              & 6.4 \\ \hline
	tqdm                 & 4.65.2 \\ \hline
	traitlets            & 5.14.1 \\ \hline
	triton               & 2.2.0 \\ \hline
	tsfresh              & 0.20.3 \\ \hline
	tsinterpret          & 0.4.7 \\ \hline
	tslearn              & 0.6.3 \\ \hline
	typing-extensions    & 4.9.0 \\ \hline
	tzdata               & 2023.4 \\ \hline
	urllib3              & 2.2.0 \\ \hline
	wcwidth              & 0.2.13 \\ \hline
	webencodings         & 0.5.1 \\ \hline
	werkzeug             & 3.0.1 \\ \hline
	wfdb                 & 4.1.2 \\ \hline
	wrapt                & 1.14.1 \\ \hline
	xarray               & 2024.3.0 \\ \hline
	xmljson              & 0.2.1 \\ \hline
	yarl                 & 1.9.4 \\ \hline
\end{longtable}
%\include{Apendices/appendixC}
%\include{...}
%\include{...}
%\include{...}
\backmatter



%
% Índice de palabras
%

% Sólo  la   generamos  si  está   declarada  \generaindice.  Consulta
% TeXiS.sty para más información.

% En realidad, el soporte para la generación de índices de palabras
% en TeXiS no está documentada en el manual, porque no ha sido usada
% "en producción". Por tanto, el fichero que genera el índice
% *no* se incluye aquí (está comentado). Consulta la documentación
% en TeXiS_pream.tex para más información.
\ifx\generaindice\undefined
\else
%\include{TeXiS/TeXiS_indice}
\fi

%
% Lista de acrónimos
%



% Sólo  lo  generamos  si  está declarada  \generaacronimos.  Consulta
% TeXiS.sty para más información.
\ifx\release\undefined
\else
	\newcommand{\acr}[2]{%
	\noindent\textbf{#1}\dots.\dotfill #2\\ \\
	}

	\newpage
	\thispagestyle{empty}
	\begin{large}
		\section*{Lista de acrónimos}
		\addcontentsline{toc}{chapter}{Lista de acrónimos}
		\acr{STFT}{Transformada de Fourier de tiempo reducido (\emph{Short-Time Fourier Transform})}
		\acr{CWT}{Transformada de onda continua (\emph{Continuous Wavelet Transform})}
		\acr{ECG}{Electrocardiograma}
		\acr{IA}{Inteligencia Artificial}
		\acr{OMS}{Organización Mundial de la Salud}
	\end{large}	
	
\fi



\ifx\generaacronimos\undefined
\else
\include{TeXiS/TeXiS_acron}
\fi

%
% Final
%
\include{Cascaras/fin}
\end{document}
